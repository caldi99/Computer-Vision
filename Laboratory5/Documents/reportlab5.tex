\documentclass{article}
\usepackage[utf8]{inputenc}
\usepackage{graphicx} %package to manage images
\graphicspath{ {./images/} }
\usepackage{sectsty}

\sectionfont{\LARGE}
\subsectionfont{\large}

\begin{document}
\begin{flushleft}
\section*{Report Lab expierience 5\\ Francesco Caldivezzi \\ ID Number : 2037893}

\subsection*{Experience Gained}
In this lab I understand how to :
\begin{itemize}
  \item Use the function \textbf{cv::threshold()} in order to apply otsu's optimal method 
  \item Use \textbf{cv::kmeans()} function and how to actually, apply k-means clustering on an image
\end{itemize}

\newpage
\subsection*{Unexpected Issues}
The main difficulties of this Lab were :
\begin{itemize}
  \item Task1 : the main difficulty here was to find a method for segmenting the 3 asphalt images that was able to "generalize". In order to achieve this, I have applied otsu's optimal method but with some pre and post processing.
  \item Task2 : the main difficulty here instead was to visualize the result of the application o k-means on an image.
  \item Task3 : again, the main difficulty as in task 1 was deciding the method to use to solve this problem. In this case I have simply apply a global threshold with some degree's of freedom in order to detect the T-shirts of the robot's players.
\end{itemize}

\newpage
\subsection*{Results}
In this section we talk about the experimental results :
\begin{itemize}
  \item Task1 : \\
  Image 1:\\
  \includegraphics[scale = 0.35]{images/task1_img1.PNG}\\
  Image 2:\\
  \includegraphics[scale = 0.35]{images/task1_img2.PNG}\\
  Image 3:\\
  \includegraphics[scale = 0.25]{images/task1_img3.PNG}\\
  \item Task 2 : \\
  Grayscale Image :\\
  \includegraphics[scale = 0.35]{images/task2_gray.PNG}\\
  Color Image :\\
  \includegraphics[scale = 0.35]{images/task2_color.PNG}\\
  \item Task 3 : \\
  \includegraphics[scale = 0.35]{images/task3.PNG}
  
  
\end{itemize}

\end{flushleft}
\end{document}
